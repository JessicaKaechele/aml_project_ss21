\section{Introduction}
\comment{Written by }

The COVID-19 pandemic has been and is still a highly relevant topic concerning all of humanity. A lot of progress towards preventing its spread has been made of the last two years. But, while a vaccine is readily available in some countries, others are struggling to get a sufficient supply. And even if the vaccine was available everywhere, not every person is medically able to receive it. It is therefore important to keep researching and developing alternative methods of detection and prevention to keep people safe. New and improved machine learning based techniques focusing on different aspects of the disease are released at a steady pace, with one such field of study being the early and easy detection of COVID from CT- or X-ray-scans to aid medical diagnosis in distinguishing it from normal pneumonia.

Privacy, and data privacy in particular, is another topic that, while it has been important for many decades, gets more attention and is put more into focus of peoples minds in recent years. Through the means of the internet, society as a whole releases more and more, often personal, information into the public domain. Massive amount of data is usually needed and can be used for good by machine learning researches. But it can also be misused and hurt those it comes from when used with the wrong intentions. Leaks of compromised private data which are then sold become more and more common the more data can be found when attacking a single big target aggregating this data. Even lists of only the worst data breaches are counting more than 40 entries for 2021 alone.\cite{data_breaches}
Medical data especially is usually regarded as highly sensitive and its storage or transfer is handled with a more careful approach. It is a reasonable request that people don't want their medical records leaked to the public. However, fast and accurate research into COVID with machine learning is difficult if not enough data is available. 

To achieve both data privacy and accurate research, the concept of federated learning could potentially help. By distributing the training of machine learning models onto multiple clients there is no need to share data between them and other entities. As it is a relatively new approach, not many COVID research papers exist that utilize it, making it hard to evaluate whether or not it is applicable in this field. We want to take some papers working with image data, replicate their methods and results and apply federated learning to the approach to analyse the positive or negative impact in performance. In addition, we want to develop a means to automate this kind of test as much as possible for approaches in other papers. 

In this report we will explain the theory of federated machine learning and the methods used by papers we used to test the federated learning approach with. We will then describe both the training of original and the federated process for each of these papers and discuss all of our results in relation to the original authors results. In the end there will be a brief summary of what we achieved and an outlook looking into improvements that could be done in the future.

Federated learning in general can also be used in other areas of research where privacy and computational power are a matter of concern.
