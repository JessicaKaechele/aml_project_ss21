\section{Conclusion and Outlook}
This section will give a brief conclusion of results that we achieved and explore some ideas and concepts that can be used to further improve our automated federation test in the future.  

\subsection{Conclusion}
We were able to replicate results of all but one of the papers that we selected and successfully adopted their approaches to utilize the concept of federated learning. In almost all cases the federated version achieved similar results to the sequential one, indicating that federated learning can indeed be used in Covid research to preserve data privacy of sensitive medical data. This could enable research projects with multiple entities, for example government health organizations, working together on one big machine learning model, which seemed to be impossible due to privacy concerns in the past.

Further, we developed a python class which can be used to quickly and with relative ease test how federated learning performs with a given model and training methodology before having to commit to building a custom solution or integrate a complex framework into existing code for that purpose. This will hopefully make the barrier-of-entrance to more development with federated learning easier and result in more projects using it in the future.

\subsection{Outlook}
Even though our automated federation test works quite well on the models we tested it with we cannot guarantee that it works on any other. It should be tested with a lot more models and datasets to get a higher confidence in its ability to indeed work with most image based Pytorch models.  
Apart from more testing there is also a lot of room for potential improvement in other areas. First, the current data augmentations for increasing the data available on each client needs to be more configurable and make more augmentations available to fit more scenarios. Taking this a step further, we could also support non-image data and their transformations.
By supporting Pytorch models we, at least in theory, also support models of frameworks that are based on Pytorch. While this is a big portion of all machine learning models, it is by no means all of them. Support for Tensorflow and its respective dataset format is almost a must if we want to make quick testing of federated learning available to a large portion of the machine learning community.

mehr FedAvg methods? (e.g. weighted, ...)
split data differently
