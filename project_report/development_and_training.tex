\section{Development \& Training}
describe training and datasets for all models

\subsection{Custom ResNet}
- for testing / as replacement for original paper
- PySyft

zuerst paper angeschaut, dabei dsa problem gehabt, dass netzwerk nicht ausreichend beschrieben wurde
-> selbst eins gebaut und uns an die angaben so nah wie möglich gehalten

duplication, shuffling, batching

\subsection{Adapted Papers}

\paragraph{COVIDNet}
\paragraph{DLH-Covid}
\paragraph{DarkCovidNet}
\paragraph{GraphCovidNet}

erst andere paper nachgebaut
deren training prozess beschreiben - vergleich ergebnisse mit paper (in anderem kapitel!)
federated process 
  -> original paper nachstellen
  -> erst mit duplizierten daten
  -> dann mit split / augmented
  -> andere paper
  -> automatisiert


\subsection{Other Papers}


\subsection{Automated Federated Learning}
automation
same gpu resources as single model (but potentially more ram)
difference in time to train not taken into consideration because the communication overhead will most likely be way more noticeable (could be studied in the future)
