\section{Development \& Training}\label{sec:dev_and_training}
\comment{Written by Jessica Kächele}
The following section will describe the development and training process of all models introduced in section \ref{subsec:methods_original_paper} and \ref{subsec:methods_adapted_papers}, including any steps needed for the respective dataset creation. 
To get the most accurate comparison against the original papers results, while developing we took the authors code that was released together with each paper and tried to get it running ourselves with the least amount of modifications to data loading, training and testing methods as possible. More detail on how each model was adapted can be found in the following subsections.
All experiments were done with either PyTorch or Tensorflow and all systems used had at least 12GB of RAM and a CUDA capable GPU equipped. For our custom ResNet training we used our own machines, while development on all the other models was done on Google Colab\footnote{\url{https://colab.research.google.com/}}, because most of the time local resources were insufficient for either data generation or training or both.  


\subsection{Custom ResNet}\label{subsec:methods_resnet}
\subcomment{Written by Jessica Kächele}
As described in section \ref{subsec:methods_original_paper} we were not able to exactly rebuild the \textit{federation papers} original model.
Instead, we took the dataset mentioned by the authors, resized the images therein to 244 by 244 pixels and treated non-disease and pneumonia images as the same class. 
We then trained a pre-built (but not pre-trained) ResNet18 architecture from PyTorch with an Adam optimizer and an instance of the \textit{BCEWithLogitsLoss} criterion for 25 epochs and a batch size of 32. The model accuracy was tested after each epoch.

Once this procedure resulted in accuracies close to the values from the paper, we started experimenting with federated learning. In the beginning we used the PySyft\footnote{Federated learning framework which has both a PyTorch and a Tensorflow backend. Available at \url{https://github.com/OpenMined/PySyft}.} framework to test what kind of results federated learning can achieve with an already established implementation, but did not get it to work correctly. The accuracy of the federated model stayed the same during all training epochs and the model did not seem to learn anything.

We quickly moved on, as we did not want to spent too much time debugging code that was not going to be used for the actual project research anyway. The first step for our own implementation was the creation of one global and $N$ local models, each with its own copy of the whole training dataset.
The next step was implementing methods to distribute the global model to the clients, which was as simple as loading the complete state of the global model by each of the local models. The FedAvg algorithm was implemented by iterating through the parameters of all local models manually and adding these individual values, each divided by the number of client models, to one dictionary. This dictionary containing the averaged model parameters was then loaded by the global model.
At the beginning of each epoch, the client models were updated with the parameters of the global model and then trained. Once each client model completed one round of training, the global model was updated.
As the Adam optimizer is initialized with its respective models parameters and the parameters of all models changed after the global model was updated, a new optimizer instance had to be created for each client after each epoch.
Also, generally, only a subset of all clients is used for training in each round, because there are diminishing returns after a certain number of clients and the communication overhead is not worth the additional accuracy.\cite{fl_paper}. Regardless of this, we use all clients for training because we only instantiate a small number of them (usually 4) and the communication overhead in our simulated federal learning setting is not as bad as in real world scenarios.

The last change that we made to our ResNet model training was using our automated federation learning class, which is illustrated in more detail in section \ref{subsec:methods_automated_fl}, with different parameters. All results discussed from this model were obtained by training with this automation.

\subsection{COVID-Net}\label{subsec:dev_covidnet}
\subcomment{Written by Jonas Reinwald}
To set up binary classification we were able to mostly take the relevant code files from the papers GitHub repository\footnote{\url{https://github.com/lindawangg/COVID-Net/}} and directly use them in a Jupyter notebook. As the authors final model was adopted from a pre-trained existing model and then modified using generative synthesis, the authors do not know how their model trained from randomly initialized weights and biases actually performs, as they have not done so themselves. Instead they provide their finished model as a Numpy array and code to re-train this model on new data. 

Creating the necessary dataset required following a five step manual with lots of manual downloads and code modifications. Creating and then uploading it for use in Google Colab took a long time, which eventually turned out to be wasted. When the dataset generation was done, we tried re-training the existing model on this data with the code published by the authors, but it performed worse and worse the more epochs it was trained, regardless of the parameter combinations we used. Because of this, we were not able to replicate the papers results and never implemented federated training with this approach. This implementation was the only one using Tensorflow as a backend, which would have made the insights gained from a federated version even more valuable from another PyTorch model.

\newpage
\subsection{DLH-COVID}
\subcomment{Written by Jessica Kächele}
Just like with the previous model we managed to get the code from the papers GitHub repository\footnote{\url{https://github.com/soudey123/COVID-19-CHEST-X-RAY-IMAGE-CLASSIFICATION_UIUC/}} running with minimal modifications, which consisted of adapting drive paths to the dataset and removing parts that were of no interest to our research.
For the federated learning, we changed the existing training functions to include multiple client models and the same FedAvg function that we used for our ResNet training. Some variables and function calls had to be adapted to work with multiple models, but the changes were not too difficult. After training, the global model could be evaluated and tested with the exact same functions as the non-federated model previously.  

\subsection{DarkCovidNet}
\subcomment{Written by Jessica Kächele}
The official DarkCovidNet implementation uses the computer vision module of a neural network training framework named fastai\footnote{\url{https://fastai1.fast.ai/index.html}}. When using this framework, the actual training and testing is not exposed in the code, which is why we decided to implement our own version of the model according to the layer information taken from the paper. For training and testing we were able to re-use the functions used for our ResNet model. Additionally, the DarkCovidNet model was trained with k-fold cross-validation using 3 folds.
As we developed this adaptation at the same time as the ResNet model, we also tried to use PySyft on it. The results were the same as before, where the model did not seem to actually train or learn anything.
We then decided to not implement federated learning manually but rather wait until the automation class was ready and use this implementation as an additional test for the automation. The final results of this model were obtained with the help of this automated process.

\subsection{GraphCovidNet}
\subcomment{Written by Jonas Reinwald}
Once again most of the implementation of this architecture was taken from the GitHub repository\footnote{\url{https://github.com/debadyuti23/GraphCovidNet/}} and modified to run in a Google Colab notebook. 
The authors provide two scripts for generating the graph data set from images. They include the option to resize images to a smaller size to reduce system memory footprint, but we were worried that we could not reproduce the papers results if we had chosen to use this option. Not using it, however, meant that the memory requirements exceeded both the amount that Google Colab makes available for free and that we have available on our local machines. We temporarily used a Google Colab Pro subscription for generating the datasets.
Training and evaluating the model on the other hand was fairly straightforward and only some minor changes had to be made to fit our environment. Integrating federated learning was more of a challenge than with the papers because the published code was not very modular and uses a lot of global variables.
At first, we only created the dataset and code for running a binary classification, but towards the end of the project a second notebook with 4-class classification was implemented as well.

\subsection{Automated Federated Learning}\label{subsec:methods_automated_fl}
\subcomment{Written by Jonas Reinwald}
One part of the goal for this project was to create a function or class, that given a PyTorch model and some more parameters, could then test the model with both centralized and federated learning. This is intended to be used as an easy way to gauge if it is worth implementing a model with a full federated learning framework. 

We started developing the automation by creating a function which can split data that is passed in the form of a PyTorch \textit{Dataset} to a list of \textit{DataLoader} instances that can then be used for the training of local client models. In addition to the \textit{Dataset} the user needs to specify the number of clients, the local batch size, whether or not data should be augmented before being passed to the client, whether or not a client gets a copy of the full dataset or only a chunk with the size $\frac{length\_of\_data}{number\_of\_clients}$, as well as some supplementary arguments which get passed to the \textit{DataLoader} instances directly.
If the data should be augmented, a composed PyTorch transformation is created which combines a random horizontal flip (50\% probability), a random vertical flip (same probability), a random rotation of up to 90 degrees in either direction and a random translation on both axis of up to 20\% of the image size.
When using full data on each client, we simply return one \textit{DataLoader} with the full dataset for each client. Otherwise, we split the dataset into random but equally sized subsets and then create \textit{DataLoader} instances with these. 
Because this function takes a \textit{Dataset} instance which can already contain transforms we had to implement our own sub-class of the \textit{Dataset} type which can hold the original dataset and an additional transformation object. Every time an item is retrieved from this new type of dataset, we take the item from the wrapped dataset and manually apply the transformations described above.

This function, together with the client update and FedAvg methods from section \ref{subsec:methods_resnet}, was then used inside the \textit{FederatedLearningTest} class, whose initialization method can be seen in listing \ref{lst:automated_fl_init}. For obvious reasons the class needs to be passed the model that should be tested and both a training as well as a testing dataset. As we do not know and more importantly do not want to control how to train (one epoch at a time) and how to test the model, functions for both of these operations need to be passed in as well. These functions need to have a certain signature in order for us to call them correctly. The train function needs to take the model, a training \textit{DataLoader}, an optimizer and a criterion as arguments and not return anything. The test function also needs to take the model, the test \textit{DataLoader} and a criterion as arguments. It can also return any printable object, we print this return value every time the test function is called internally.
Further initialization arguments for the class include whether or not to use the GPU, how many global epochs or rounds should be trained and how many local epochs should be trained. It was important for us to never use more GPU resources for the complete comparison than is needed for training just the centralized model. Additional system memory may be necessary, but it is also easier and cheaper to equip a system with.

\begin{lstlisting}[language=CustomPython, style=colorEX, caption=Automated federation learning class initialization, captionpos=t, label=lst:automated_fl_init]
  class FederatedLearningTest:
    def __init__(
      self,
      model: torch.nn.Module,
      train_dataset: torch.utils.data.Dataset,
      test_dataset: torch.utils.data.Dataset,
      train_epoch_fn: Callable[
        [torch.nn.Module, torch.utils.data.DataLoader, torch.optim.Optimizer, torch.nn.modules.loss._Loss], 
        None
      ], 
      test_fn: Callable[
        [torch.nn.Module, torch.utils.data.DataLoader, torch.nn.modules.loss._Loss],
        Any
      ],
      use_gpu: bool,
      epochs_to_train: int,
      local_epochs_to_train: int
    ): [...]
\end{lstlisting}

Once an instance of the class is created, the compare method, the signature of which can be seen in listing \ref{lst:automated_fl_compare}, can be called with different parameters as many times as needed. The first two arguments are only used for preparing and splitting the data as portrayed above. The number of clients is also used for data preparation in addition to creating the correct amount of client models and looping over them. The compare function also needs to be passed two functions which return a new PyTorch optimizer and a PyTorch loss or criterion instance respectively, because these have to be constructed more than once during the training process. The last argument controls whether the model is tested after every epoch or only once at the end of training.

\begin{lstlisting}[language=CustomPython, style=colorEX, caption=Automated federation learning compare method input, captionpos=t, label=lst:automated_fl_compare]
  def compare(
    self,
    augment_data: bool,
    full_data_on_each_client: bool,
    no_of_clients: int,
    construct_optimizer_fn: Callable[
      [torch.nn.Module],
      torch.optim.Optimizer
    ],
    construct_loss_fn: Callable[
      [],
      torch.nn.modules.loss._Loss
    ],
    test_after_each_epoch: bool = False
  ): [...]
\end{lstlisting}

When calling the compare method, we begin with training of the non-federated model. A deep copy of the initial model, an optimizer and a criterion as well as train and test \textit{DataLoader}s are created. The model is then trained for the specified number of epochs by calling the given train function in a loop. After training is complete, the test function is called in the same way and its result gets printed. Our implementation completely automates moving of the model and data to and from the GPU as well as setting the model to the appropriate mode for training and testing without any help from the user. 
The same process is then repeated for federated learning, this time with one shared  and $ number\_of\_clients $ local models copied from the initial one. Next, the data is prepared according to the arguments that were passed. The training process contains one additional loop to go over each client model as well as the model parameter updating between clients and global model, but is otherwise the same as before.
In the end, results from both training instances are printed for the user to consume and evaluate.