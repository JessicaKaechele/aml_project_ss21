\section{Development \& Training}\label{sec:dev_and_training}
describe training and datasets generation for all models

% TODO from results: As mentioned in section \ref{sec:dev_and_training} we did not want to change more than absolutely necessary from the original implementation, including the testing and evaluation methods, which is why the presentation and form of results can differ from one paper to another.

\subsection{Custom ResNet}
- for testing / as replacement for original paper
- PySyft

zuerst paper angeschaut, dabei dsa problem gehabt, dass netzwerk nicht ausreichend beschrieben wurde
-> selbst eins gebaut und uns an die angaben so nah wie möglich gehalten

duplication, shuffling, batching

\subsection{Adapted Papers}

\subsubsection{COVID-Net}\label{subsubsec:dev_covidnet}

\subsubsection{DLH-COVID}
\subsubsection{DarkCovidNet}
Model kopiert, Rest selbst implementiert, weil: mit PySyft testen und dann als Test für automated federation nutzen
\subsubsection{GraphCovidNet}


The graph generation described in section \ref{subsubsec:methods_graphcovidnet} consumed a lot of system memory. 

erst andere paper nachgebaut
deren training prozess beschreiben - vergleich ergebnisse mit paper (in anderem kapitel!)
federated process 
  -> original paper nachstellen
  -> erst mit duplizierten daten
  -> dann mit split / augmented
  -> andere paper
  -> automatisiert


\subsection{Other Papers}


\subsection{Automated Federated Learning}
automation
same gpu resources as single model (but potentially more ram)
difference in time to train not taken into consideration because the communication overhead will most likely be way more noticeable (could be studied in the future)
