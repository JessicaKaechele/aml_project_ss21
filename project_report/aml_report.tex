% This template was initially provided by Dulip Withanage.
% Modifications for the database systems research group
% were made by Conny Junghans,  Jannik Strötgen and Michael Gertz

\documentclass[
     12pt,         % font size
     a4paper,      % paper format
     BCOR10mm,     % binding correction
     DIV14,        % stripe size for margin calculation
     ]{article}

%%%%%%%%%%%%%%%%%%%%%%%%%%%%%%%%%%%%%%%%%%%%%%%%%%%%%%%%%%%%

% PACKAGES:

% Use German :
\usepackage[english]{babel}
% Input and font encoding
\usepackage[utf8]{inputenc}
\usepackage[T1]{fontenc}
\usepackage[title]{appendix}
% Index-generation
\usepackage{makeidx}
% Einbinden von URLs:
\usepackage{url}
% Special \LaTex symbols (e.g. \BibTeX):
%\usepackage{doc}
% Include Graphic-files:
\usepackage{graphicx}
% Include doc++ generated tex-files:
%\usepackage{docxx}

% Fuer anderthalbzeiligen Textsatz
\usepackage{setspace}

\usepackage{amsmath}
\usepackage{amssymb}
% F�r todos
% \usepackage[obeyFinal]{easy-todo}

% hyperrefs in the documents
\PassOptionsToPackage{hyphens}{url}\usepackage[
  bookmarks=true,
  colorlinks,
  pdfpagelabels,
  pdfstartview = FitH,
  bookmarksopen = true,
  bookmarksnumbered = true,
  linkcolor = black,
  plainpages = false,
  hypertexnames = false,
  citecolor = black,
  urlcolor=black]{hyperref}
%\usepackage{hyperref}

% Custom colors
\usepackage{color}
\definecolor{darkred}{rgb}{0.6,0.0,0.0}
\definecolor{darkgreen}{rgb}{0,0.50,0}
\definecolor{lightblue}{rgb}{0.0,0.42,0.91}
\definecolor{orange}{rgb}{0.99,0.48,0.13}
\definecolor{grass}{rgb}{0.18,0.80,0.18}
\definecolor{pink}{rgb}{0.97,0.15,0.45}

\usepackage{csquotes}
\usepackage{tabularx}
\usepackage{listings}

% General Setting of listings
\lstset{
  aboveskip=1em,
  breaklines=true,
  abovecaptionskip=-6pt,
  captionpos=b,
  escapeinside={\%*}{*)},
  frame=single,
  numbers=left,
  numbersep=15pt,
  numberstyle=\tiny,
}
% 0. Basic Color Theme
\lstdefinestyle{colored}{ %
  basicstyle=\ttfamily,
  backgroundcolor=\color{white},
  commentstyle=\color{green}\itshape,
  keywordstyle=\color{blue}\bfseries\itshape,
  stringstyle=\color{red},
}
% 1. General Python Keywords List
\lstdefinelanguage{PythonPlus}[]{Python}{
  morekeywords=[1]{,as,assert,nonlocal,with,yield,self,True,False,None,} % Python builtin
  morekeywords=[2]{,__init__,__add__,__mul__,__div__,__sub__,__call__,__getitem__,__setitem__,__eq__,__ne__,__nonzero__,__rmul__,__radd__,__repr__,__str__,__get__,__truediv__,__pow__,__name__,__future__,__all__,Any, Callable}, % magic methods
  morekeywords=[3]{,,object,type,isinstance,copy,deepcopy,zip,enumerate,reversed,list,set,len,dict,tuple,range,xrange,append,execfile,real,imag,reduce,str,repr,}, % common functions
  morekeywords=[4]{,Exception,NameError,IndexError,SyntaxError,TypeError,ValueError,OverflowError,ZeroDivisionError,}, % errors
  morekeywords=[5]{,ode,fsolve,sqrt,exp,sin,cos,arctan,arctan2,arccos,pi, array,norm,solve,dot,arange,isscalar,max,sum,flatten,shape,reshape,find,any,all,abs,plot,linspace,legend,quad,polyval,polyfit,hstack,concatenate,vstack,column_stack,empty,zeros,ones,rand,vander,grid,pcolor,eig,eigs,eigvals,svd,qr,tan,det,logspace,roll,min,mean,cumsum,cumprod,diff,vectorize,lstsq,cla,eye,xlabel,ylabel,squeeze,}, % numpy / math
}
% 2. New Language based on Python
\lstdefinelanguage{CustomPython}[]{PythonPlus}{
  morekeywords=[2]{,Any}, 
  morekeywords=[3]{,Dataset, Callable, DataLoader, Module, Optimizer, _Loss},
  emph={FederatedLearningTest, compare},
}
% 3. Extended theme
\lstdefinestyle{colorEX}{
  basicstyle=\footnotesize\ttfamily,
  backgroundcolor=\color{white},
  commentstyle=\color{darkgreen}\slshape,
  keywordstyle=\color{blue}\bfseries\itshape,
  keywordstyle=[2]\color{blue}\bfseries,
  keywordstyle=[3]\color{grass},
  keywordstyle=[4]\color{red},
  keywordstyle=[5]\color{orange},
  stringstyle=\color{darkred},
  emphstyle=\color{pink}\underbar,
}

% subfigures
\usepackage{subcaption}
\usepackage[font={small,it}]{caption}


%%%%%%%%%%%%%%%%%%%%%%%%%%%%%%%%%%%%%%%%%%%%%%%%%%%%%%%%%%%%

% OTHER SETTINGS:

% Choose language
\newcommand{\setlang}[1]{\selectlanguage{#1}\nonfrenchspacing}

% Written by comment
\newcommand{\comment}[1]{\vspace{-1em}\hspace{27pt}{\small\textit{#1}}\bigskip\par}
\newcommand{\subcomment}[1]{\vspace{-0.8em}\hspace{35pt}{\small\textit{#1}}\bigskip\par}
\newcommand{\subsubcomment}[1]{\vspace{-0.8em}\hspace{39pt}{\small\textit{#1}}\bigskip\par}
\newcommand{\paragraphcomment}[1]{\hspace{-8pt}{\small\textit{#1}}\hspace{8pt}}

\DeclareMathOperator*{\argmax}{arg\,max}
\DeclareMathOperator*{\argmin}{arg\,min}

\setcounter{tocdepth}{2} 

\begin{document}

% TITLE:
\pagenumbering{roman} 
\begin{titlepage}


\begin{center}
\textbf{ 
\Large Heidelberg University\\
\smallskip
}

\vspace{2cm}

\textbf{\large Project Report - Advanced Machine Learning}

\vspace{0.5\baselineskip}
{\huge
\textbf{Federated Machine Learning}
}
\end{center}

\vfill 

{\large
\begin{tabular}[l]{ll}
Team Member: & Jessica Kaechele, 3588787,\\ & MSc Applied Computer Science\\
  & Uo251@stud.uni-heidelberg.de\\
Team Member: & Jonas Reinwald, 3600238, \\ & MSc Applied Computer Science\\
  & am248@stud.uni-Heidelberg.de\\
\end{tabular}
}
\vspace*{2cm}

{
  \begin{center}
  \textbf{GitHub Repository:}
  \url{https://github.com/JessicaKaechele/aml_project_ss21}
  \end{center}
}

\vspace*{.5cm}

\end{titlepage} 

% \input{<file>}
\newpage
\tableofcontents

\newpage
\listoffigures
\listoftables
\lstlistoflistings

\newpage

\pagenumbering{arabic}

\begin{abstract}
  Both COVID-19 and data privacy are without a doubt two highly relevant topics of our time.
  There is a lot of COVID-19 research happening in the area of machine learning that generally requires large amount of data.
  This often leads to hard decisions between preserving personal medical data privacy and not having enough data to do research and vice versa.
  With Federated Learning there exists a paradigm which can combine both of these needs.
  In this project, which is part of the Advanced Machine Learning lecture given by Prof. Dr. Ullrich Köthe at the University of Heidelberg\cite{aml_lecture}, we explore if machine learning COVID-19 research can be combined with federated learning to preserve data privacy and how it affects the performance of already existing machine learning models.
  We also present a way to test such models with federated learning while requiring almost no change to existing code in the hope that federated learning will be adopted by more researchers in the future.
  Our findings indicate that machine learning architectures used for COVID-19 research can indeed be adopted to use federated machine learning with low to no negative impact on their accuracy.
\end{abstract}

\section{Introduction}
\comment{Written by }

The Covid-19 pandemic has been and is still a highly relevant topic concerning all of humanity. A lot of progress towards preventing its spread has been made of the last two years. But, while a vaccine is readily available in some countries, others are struggling to get a sufficient supply. And even if the vaccine was available everywhere, not every person is medically able to receive it. It is therefore important to keep researching and developing alternative methods of detection and prevention to keep people safe. New and improved machine learning based techniques focusing on different aspects of the disease are released at a steady pace, with one such field of study being the early and easy detection of Covid from CT- or X-ray-scans to aid medical diagnosis in distinguishing it from normal pneumonia.

Privacy, and data privacy in particular, is another topic that, while it has been important for many decades, gets more attention and is put more into focus of peoples minds in recent years. Through the means of the internet, society as a whole releases more and more, often personal, information into the public domain. Massive amount of data is usually needed and can be used for good by machine learning researches. But it can also be misused and hurt those it comes from when used with the wrong intentions. Leaks of compromised private data which are then sold become more and more common the more data can be found when attacking a single big target aggregating this data. Even lists of only the worst data breaches are counting more than 40 entries for 2021 alone.\cite{data_breaches}
Medical data especially is usually regarded as highly sensitive and its storage or transfer is handled with a more careful approach. It is a reasonable request that people don't want their medical records leaked to the public. However, fast and accurate research into Covid with machine learning is difficult if not enough data is available. 

To achieve both data privacy and accurate research, the concept of federated learning could potentially help. By distributing the training of machine learning models onto multiple clients there is no need to share data between them and other entities. As it is a relatively new approach, not many Covid research papers exist that utilize it, making it hard to evaluate whether or not it is applicable in this field. We want to take some papers working with image data, replicate their methods and results and apply federated learning to the approach to analyse the positive or negative impact in performance. In addition, we want to develop a means to automate this kind of test as much as possible for approaches in other papers. 

In this report we will explain the theory of federated machine learning and the methods used by papers we used to test the federated learning approach with. We will then describe both the training of original and the federated process for each of these papers and discuss all of our results in relation to the original authors results. In the end there will be a brief summary of what we achieved and an outlook looking into improvements that could be done in the future.

Federated learning in general can also be used in other areas of research where privacy and computational power are a matter of concern.

\section{Methods}

no actual split of clients (only virtual as different models in same machine)

\subsection{Federated Learning}


\subsection{Paper replication}

\subsection{Methods of papers}

methods of other papers (focus on pytorch image data)
federated Learning
(maybe only in outlook) tensorflow-federated erwähnen?


\section{Development \& Training}\label{sec:dev_and_training}
The following section will describe the development and training process of all models introduced in section \ref{subsec:methods_original_paper} and \ref{subsec:methods_adapted_papers}, including any steps needed for the respective dataset creation. 
To get the most accurate comparison against the original papers results, while developing we took the authors code that was released together with each paper and tried to get it running ourselves with the least amount of modifications to data loading, training and testing methods as possible. More detail on how each model was adapted can be found in the following subsections.
All experiments were done with either PyTorch or Tensorflow and all systems used had at least 12GB of RAM and a CUDA capable GPU equipped. For our custom ResNet training we used our own machines, while development on all the other models was done on Google Colab\footnote{\url{https://colab.research.google.com/}}, because most of the time local resources were insufficient for either data generation or training or both.  


\subsection{Custom ResNet}\label{subsec:methods_resnet}
As described in section \ref{subsec:methods_original_paper} we were not able to exactly rebuild the \textit{federation papers} original model.
Instead, we took the dataset mentioned by the authors, resized them to 244 by 244 pixels and treated non-disease and pneumonia images as the same class. 
We then trained a pre-built (but not pre-trained) ResNet18 from PyTorch with an Adam optimizer and an instance of the \textit{BCEWithLogitsLoss} criterion for 25 epochs and a batch size of 32. The model accuracy was tested after each epoch.

Once this procedure resulted in accuracies close to the values from the paper, we started experimenting with federated learning. In the beginning we used the PySyft\footnote{Federated learning framework which can has both a PyTorch and a Tensorflow backend. Available at \url{https://github.com/OpenMined/PySyft}.} framework to test which results federated learning can achieve with an already established implementation, but did not get it to work correctly. The accuracy of the federated model stayed the same during all training epochs and the model did not seem to learn anything, most likely caused by a user-error regarding PySyft.

We quickly moved on, as we did not want to spent too much time debugging code that was not going to be used for the actual project research anyway. The first step for our own implementation was the creation of one global and $N$ local models, each with its own copy of the whole training dataset.
The next step was implementing methods to distribute the global model to the clients, which was as simple as loading the complete state of the global model by each of the local models. The FedAvg algorithm was implemented by iterating through the parameters of all local models manually and adding these individual values, each divided by the count of client models, to one dictionary. This dictionary containing the averaged model parameters was then loaded by the global model.
At the beginning of each epoch, the client models were updated with the parameters of the global model and then trained. Once each client model completed one round of training, the global model was updated.
As the Adam optimizer is initialized with its respective models parameters and the parameters of all models changed after the global model was updated, a new optimizer instance had to be created for each client after each epoch.
Also, generally, only a subset of all clients is used for training in each round, because there are diminishing returns after a certain number of clients and the communication overhead is not worth the additional accuracy.\cite{fl_paper}. Regardless of this, we use all clients because we only use a small number of them (usually 4) and the communication overhead is not as bad in our simulated federal learning setting than in a real world scenario.

The last change that we made to our ResNet model training was using our automated federation learning class, which is illustrated in more detail in section \ref{subsec:methods_automated_fl}, with different parameters. All results discussed from this model were obtained by training with this automation.

\subsection{COVID-Net}\label{subsec:dev_covidnet}
To set up binary classification we were able to mostly take the relevant code files from the papers GitHub repository\footnote{\url{https://github.com/lindawangg/COVID-Net/}} and directly use them in a notebook. As the authors final model was adopted from a pre-trained existing model and then modified using generative synthesis, the authors do not know how to actually train it from randomly initialized weights and biases. Instead they provide their finished model as a Numpy array and only provide code to re-train this model on new data. 

Generating the necessary dataset required following a five step manual with lots of manual downloads and code modifications. Creating and then uploading it for use in Google Colab took a long time, which eventually turned out to be wasted. When this step was done, we tried re-training the existing model on the generated dataset with exactly the code provided by the authors, but it performed worse the more epochs it was trained with every combination of parameters that we used. Because of this, we were not able to replicate the papers results and never implemented federated training with this approach. This implementation was the only one using Tensorflow as a backend, which would have made the insights gained from it even more valuable.

\subsection{DLH-COVID}
Just like with the previous model we managed to get the code from the papers GitHub repository\footnote{\url{https://github.com/soudey123/COVID-19-CHEST-X-RAY-IMAGE-CLASSIFICATION_UIUC/}} running with minimal modifications, which consisted of adapting drive paths to the dataset and removing parts that were of no interest to our research.
For the federated learning, we changed the existing training functions to include multiple client models and the same FedAvg function that we used for our ResNet training. Some variables and function calls had to be adapted to work with multiple models, but the changes were not too difficult. After training, the global model could be evaluated and tested with the exact same functions as the non-federated model previously.  

\subsection{DarkCovidNet}
The official DarkCovidNet implementation uses the computer vision module of a neural network training framework named fastai\footnote{\url{https://fastai1.fast.ai/index.html}}. Using this framework, the actual training and testing is not exposed in the code, which is why we decided to implement our own version of the model according to the layer information taken from the paper. For training and testing we were able to re-use the functions used for our ResNet model. Additionally, the DarkCovidNet model was trained with k-fold cross-validation using 3 folds.
As we developed this adaptation at the same time as the ResNet model, we also tried to use PySyft on it. The results were the same as before, where the model did not seem to actually train or learn anything.
We then decided to not implement federated learning manually but rather wait until the automation class was ready to test it with an additional model. The final results of this model were obtained with the help of this automated process.

\subsection{GraphCovidNet}
Once again most of the implementation of this architecture was taken from the GitHub repository\footnote{\url{https://github.com/debadyuti23/GraphCovidNet/}} and modified to run in a Google Colab notebook. 
The authors provide two scripts for generation of the graph data set from images. They include the option to resize images to a smaller size to reduce system memory footprint, but we were worried that we could not reproduce the papers results if we had chosen to use this option. Not using it, however, meant that the memory requirements exceeded both the amount that Google Colab makes available for free and that we have available on our local machines. We temporarily used a Google Colab Pro subscription for generating the datasets.
Training and evaluating the model on the other hand was fairly straightforward and only some minor changes had to be made to fit our environment. Integrating federated learning was more of a challenge than with the papers because the code is not very modular and uses a lot of global variables.
At first, we only created the dataset and code for running a binary classification, but towards the end of the project a second notebook with 4-class classification was implemented as well.

\subsection{Automated Federated Learning}\label{subsec:methods_automated_fl}
One part of our goal for the project was to create a function or class which could be given a PyTorch model and some more parameters and test the model with both centralized and federated learning for an easy to use way to gauge if it is worth implementing a model with a full federated learning framework. 

We started developing the automation by creating a function which can split data that is passed in the form of a PyTorch \textit{Dataset} to a list of \textit{DataLoader} instances which can then be used for training of a local model. In addition to the \textit{Dataset} the user needs to specify the number of clients, the local batch size, whether or not data is augmented before being passed to the client, whether or not a client gets a copy of the full dataset or only a chunk of the size $ 1 / \text{number\_of\_clients} $ as well as some supplementary arguments which get passed to the \textit{DataLoader} instances directly.
If the data should be augmented a composed PyTorch transformation is created which include a random horizontal flip (with 50\% probability), a random vertical flip (same probability), a random rotation of up to 90 degrees in either direction and a random translation on both axis of up to 20\% of the image size.
When using full data on each client, we simply return one \textit{DataLoader} with the full dataset for each client. Otherwise, we split the dataset into random but equally sized subsets and then create \textit{DataLoader} instances with these. 
Because this function takes a \textit{Dataset} instance which can already contain transforms we had to implement our own sub-class of the \textit{Dataset} type which hold the original dataset and a transformation object. Every time an item is retrieved from this new type of dataset, we take the item from the wrapped dataset and manually apply the transformations described above.

This function, together with the client update and FedAvg methods from section \ref{subsec:methods_resnet}, was then used inside the \textit{FederatedLearningTest} class, whose initialization method can be seen in listing \ref{lst:automated_fl_init}. For obvious reasons the class needs to be passed the model that should be tested and both a training as well as a testing dataset. As we do not know and more importantly do not want to control how to train (one epoch at a time) and how to test the model, functions for both of these operations need to be passed in as well. These functions need to have a certain signature in order for us to call them correctly. The train function needs to take the model, a training \textit{DataLoader}, an optimizer and a criterion as arguments and not return anything. The test function also needs to take the model, the test \textit{DataLoader} and a criterion as arguments. It can also return any printable object, we print this return value every time the test function is called internally.
Further initialization arguments for the class include whether or not to use the GPU, how many global epochs or round should be trained and how many local epochs should be trained. 

\begin{lstlisting}[language=CustomPython, style=colorEX, caption=Automated federation learning class initialization, captionpos=t, label=lst:automated_fl_init]
  class FederatedLearningTest:
    def __init__(
      self,
      model: torch.nn.Module,
      train_dataset: torch.utils.data.Dataset,
      test_dataset: torch.utils.data.Dataset,
      train_epoch_fn: Callable[
        [torch.nn.Module, torch.utils.data.DataLoader, torch.optim.Optimizer, torch.nn.modules.loss._Loss], 
        None
      ], 
      test_fn: Callable[
        [torch.nn.Module, torch.utils.data.DataLoader, torch.nn.modules.loss._Loss],
        Any
      ],
      use_gpu: bool,
      epochs_to_train: int,
      local_epochs_to_train: int
    ): [...]
\end{lstlisting}

Once an instance of the class is created, the compare method, the signature of which can be seen in listing \ref{lst:automated_fl_compare}, can be called with different parameters as many times as needed. The first two arguments are only used for preparing and splitting the data as portrayed above. The number of clients is also used for data preparation in addition to creating the correct amount of client models and looping over them. The compare function also need to be passed two functions which return a new PyTorch optimizer and a PyTorch loss or criterion instance respectively because these have to be constructed more than once during the training process. The last argument controls whether the model is tested after every epoch or only once at the end of training.

\begin{lstlisting}[language=CustomPython, style=colorEX, caption=Automated federation learning compare method input, captionpos=t, label=lst:automated_fl_compare]
  def compare(
    self,
    augment_data: bool,
    full_data_on_each_client: bool,
    no_of_clients: int,
    construct_optimizer_fn: Callable[
      [torch.nn.Module],
      torch.optim.Optimizer
    ],
    construct_loss_fn: Callable[
      [],
      torch.nn.modules.loss._Loss
    ],
    test_after_each_epoch: bool = False
  ): [...]
\end{lstlisting}

When calling the compare method, we begin with training of the non-federated model. A deep copy of the initial model, an optimizer and a criterion as well as train and test \textit{DataLoader}s are created. The model is then trained for the specified number of epochs by calling the given train function in a loop. After training is complete the test function is called in the same way and its result gets printed. Our implementation completely automates moving of the model and data to and from the GPU and setting the model to the appropriate mode for training and testing without any help from the user.
The same process is then repeated for federated learning. Except for one copy of the initial model there are now $ 1 + \text{number\_of\_clients} $ models (1 global plus clients) copied from the initial one. Next, the data is prepared according to the arguments that were passed. The training process contains one additional loop to go over each client model as well as the model parameter updating between clients and global model, but is otherwise the same as before.
In the end, results from both training rounds are printed for the user to consume and evaluate.
\section{Exerimental Results}
This section will show and compare results obtained from original approaches, our implementation of them and their adoption to federated learning. As mentioned in chapter \ref{} we did not want to change more than absolutely necessary from the original implementation, including the testing and evaluation methods, which is why the presentation and form of results can differ from one paper to another.

\subsection{Paper 1}
\subsection{Paper 2}
\subsection{Paper 3}
\subsection{Paper 4}
\subsection{Paper 5}
vergleich ergebnisse paper vs nachgebautem Code
vergleich sequential vs federated
kurz auf automatisches Federated eingehen und welche Ergebnisse damit erzielt werden können?

\section{Conclusion and Outlook}
This section will give a brief conclusion of results that we achieved and explore some ideas and concepts that can be used to further improve our automated federation test in the future.  

\subsection{Conclusion}
We were able to replicate results of all but one of the papers that we selected and successfully adopted their approaches to utilize the concept of federated learning. In almost all cases the federated version achieved similar results to the centralized one, indicating that federated learning can indeed be used in Covid research to preserve data privacy of sensitive medical data. The adaption was successful in CNN as well as GNN architectures, showing that not only one type of network can be used with federated learning. This could enable research projects with multiple entities, for example government health organizations, working together on one big machine learning model, which seemed to be impossible due to privacy concerns in the past.
We did not deploy our federation models on different hardware and thus completely ignored the communication usually necessary for aggregating local models. In addition, we used Federated Averaging, the most basic aggregation technique in federated learning, instead of a more sophisticated approach like weighted Federated Averaging or algorithms built on top of it like \enquote{FedProx}\cite{fed_prox} or \enquote{FedSplit}\cite{fed_split}. Nevertheless, our results still show that federated learning has a huge potential for Covid and other research and that further research with it should be done.

Further, we developed a python class which can be used to quickly and with relative ease test how federated learning performs with a given model and training methodology before having to commit to building a custom solution or integrate a complex framework into existing code for that purpose. This will hopefully make the barrier-of-entrance to more development with federated learning easier and result in more projects using it in the future.

\subsection{Outlook}
Even though our automated federation test works quite well on the models we tested it with we cannot guarantee that it works on any other. It should be tested with a lot more models and datasets to get a higher confidence in its ability to indeed work with most image based Pytorch models.  
Apart from more testing there is also a lot of room for potential improvement in other areas. First, the current data augmentations for increasing the data available on each client needs to be more configurable and make more augmentations available to fit more scenarios. Taking this a step further, we could also support non-image data and their transformations. For being able to test even more real-world scenarios, it would also be of value to be able to split and distribute the data in different, non-random ways (e.g. give some clients most of the data of one class and other clients most of the data of other classes).
By supporting Pytorch models we, at least in theory, also support models of frameworks that are based on Pytorch. While this is a big portion of all machine learning models, it is by no means all of them. Support for Tensorflow and its respective dataset format is almost a must if we want to make quick testing of federated learning available to a large portion of the machine learning community.


\bibliographystyle{plain}
\bibliography{bibtex/references}

\end{document}
