% This template was initially provided by Dulip Withanage.
% Modifications for the database systems research group
% were made by Conny Junghans,  Jannik Strötgen and Michael Gertz

\documentclass[
     12pt,         % font size
     a4paper,      % paper format
     BCOR10mm,     % binding correction
     DIV14,        % stripe size for margin calculation
     ]{article}

%%%%%%%%%%%%%%%%%%%%%%%%%%%%%%%%%%%%%%%%%%%%%%%%%%%%%%%%%%%%

% PACKAGES:

% Use German :
\usepackage[english]{babel}
% Input and font encoding
\usepackage[utf8]{inputenc}
\usepackage[T1]{fontenc}
\usepackage[title]{appendix}
% Index-generation
\usepackage{makeidx}
% Einbinden von URLs:
\usepackage{url}
% Special \LaTex symbols (e.g. \BibTeX):
%\usepackage{doc}
% Include Graphic-files:
\usepackage{graphicx}
% Include doc++ generated tex-files:
%\usepackage{docxx}

% Fuer anderthalbzeiligen Textsatz
\usepackage{setspace}

\usepackage{amsmath}
\usepackage{amssymb}
% F�r todos
\usepackage[obeyFinal]{easy-todo}

% hyperrefs in the documents
\PassOptionsToPackage{hyphens}{url}\usepackage[
  bookmarks=true,
  colorlinks,
  pdfpagelabels,
  pdfstartview = FitH,
  bookmarksopen = true,
  bookmarksnumbered = true,
  linkcolor = black,
  plainpages = false,
  hypertexnames = false,
  citecolor = black,
  urlcolor=black]{hyperref}
%\usepackage{hyperref}

% subfigures
\usepackage{subcaption}


%%%%%%%%%%%%%%%%%%%%%%%%%%%%%%%%%%%%%%%%%%%%%%%%%%%%%%%%%%%%

% OTHER SETTINGS:

% Choose language
\newcommand{\setlang}[1]{\selectlanguage{#1}\nonfrenchspacing}

% Written by comment
\newcommand{\comment}[1]{\vspace{-1em}\hspace{27pt}{\small\textit{#1}}\bigskip\par}
\newcommand{\subcomment}[1]{\vspace{-0.8em}\hspace{35pt}{\small\textit{#1}}\bigskip\par}
\newcommand{\subsubcomment}[1]{\vspace{-0.8em}\hspace{39pt}{\small\textit{#1}}\bigskip\par}
\newcommand{\paragraphcomment}[1]{\hspace{-8pt}{\small\textit{#1}}\hspace{8pt}}

\DeclareMathOperator*{\argmax}{arg\,max}
\DeclareMathOperator*{\argmin}{arg\,min}

\setcounter{tocdepth}{2} 

\begin{document}

% TITLE:
\pagenumbering{roman} 
\begin{titlepage}


\begin{center}
\textbf{ 
\Large Heidelberg University\\
\smallskip
}

\vspace{2cm}

\textbf{\large Project Report - Advanced Machine Learning}

\vspace{0.5\baselineskip}
{\huge
\textbf{Federated Machine Learning}
}
\end{center}

\vfill 

{\large
\begin{tabular}[l]{ll}
Team Member: & Jessica Kaechele, 3588787,\\ & MSc Applied Computer Science\\
  & Uo251@stud.uni-heidelberg.de\\
Team Member: & Jonas Reinwald, 3600238, \\ & MSc Applied Computer Science\\
  & am248@stud.uni-Heidelberg.de\\
\end{tabular}
}
\vspace*{2cm}

{
  \begin{center}
  \textbf{GitHub Repository:}
  \url{https://github.com/JessicaKaechele/aml_project_ss21}
  \end{center}
}

\vspace*{.5cm}

\end{titlepage} 

% \input{<file>}
\newpage
\tableofcontents

\newpage
\listoffigures
\listoftables

\newpage

\pagenumbering{arabic}

\begin{abstract}
    covid and privacy both highly aktuelle topics
    covid is bla
    privacy wegen bla
    what do they have in common / how can they be combined

    We evaluate die umsetzbarkeit und performance von federated learning and present a way to test it with almost no changes to existing code required


    Research nowadays is at an all-time high, with research papers getting published every day.
    Over the last decade alone, the number of research papers on machine learning has more than tripled \cite{dimensions}.
    However, this comes with a downside, as it is getting harder and harder to keep track of trends and find specific information.
    To address this issue, we propose a clustering solution that exploits the standardized submission pipeline.
    Each submission consists of a title, an abstract, a text body, and a set of keywords that best describe the paper.
    We use the keywords from a set of over two thousand research papers on machine learning as supervision for clustering the abstracts in the hopes of finding and exploring subtopics.
    
\end{abstract}

\section{Introduction}
\comment{Written by }

The Covid-19 pandemic has been and is still a highly relevant topic concerning all of humanity. A lot of progress towards preventing its spread has been made of the last two years. But, while a vaccine is readily available in some countries, others are struggling to get a sufficient supply. And even if the vaccine was available everywhere, not every person is medically able to receive it. It is therefore important to keep researching and developing alternative methods of detection and prevention to keep people safe. New and improved machine learning based techniques focusing on different aspects of the disease are released at a steady pace, with one such field of study being the early and easy detection of Covid from CT- or X-ray-scans to aid medical diagnosis in distinguishing it from normal pneumonia.

Privacy, and data privacy in particular, is another topic that, while it has been important for many decades, gets more attention and is put more into focus of peoples minds in recent years. Through the means of the internet, society as a whole releases more and more, often personal, information into the public domain. Massive amount of data is usually needed and can be used for good by machine learning researches. But it can also be misused and hurt those it comes from when used with the wrong intentions. Leaks of compromised private data which are then sold become more and more common the more data can be found when attacking a single big target aggregating this data. Even lists of only the worst data breaches are counting more than 40 entries for 2021 alone.\cite{data_breaches}
Medical data especially is usually regarded as highly sensitive and its storage or transfer is handled with a more careful approach. It is a reasonable request that people don't want their medical records leaked to the public. However, fast and accurate research into Covid with machine learning is difficult if not enough data is available. 

To achieve both data privacy and accurate research, the concept of federated learning could potentially help. By distributing the training of machine learning models onto multiple clients there is no need to share data between them and other entities. As it is a relatively new approach, not many Covid research papers exist that utilize it, making it hard to evaluate whether or not it is applicable in this field. We want to take some papers working with image data, replicate their methods and results and apply federated learning to the approach to analyse the positive or negative impact in performance. In addition, we want to develop a means to automate this kind of test as much as possible for approaches in other papers. 

In this report we will explain the theory of federated machine learning and the methods used by papers we used to test the federated learning approach with. We will then describe both the training of original and the federated process for each of these papers and discuss all of our results in relation to the original authors results. In the end there will be a brief summary of what we achieved and an outlook looking into improvements that could be done in the future.

Federated learning in general can also be used in other areas of research where privacy and computational power are a matter of concern.

\section{Methods}

no actual split of clients (only virtual as different models in same machine)

\subsection{Federated Learning}


\subsection{Paper replication}

\subsection{Methods of papers}

methods of other papers (focus on pytorch image data)
federated Learning
(maybe only in outlook) tensorflow-federated erwähnen?


\section{Development \& Training}

\subsection{Custom Resnet}
- for testing / as replacement for original paper
- pysyft

zuerst paper angeschaut, dabei dsa problem gehabt, dass netzwerk nicht ausreichend beschrieben wurde
-> selbst eins gebaut und uns an die angaben so nah wie möglich gehalten


\subsection{Other Papers}
erst andere paper nachgebaut
deren training prozess beschreiben - vergleich ergebnisse mit paper (in anderem kapitel!)
federated process 
  -> original paper nachstellen
  -> erst mit duplizierten daten
  -> dann mit split / augmented
  -> andere paper
  -> automatisiert


\subsection{Papers not used}


\subsection{Automated Federated Learning}
automation
same gpu resources as single model (but potentially more ram)
difference in time to train not taken into consideration because the communication overhead will most likely be way more noticeable (could be studied in the future)

\section{Exerimental Results}
This section will show and compare results obtained from original approaches, our implementation of them and their adoption to federated learning. As mentioned in chapter \ref{} we did not want to change more than absolutely necessary from the original implementation, including the testing and evaluation methods, which is why the presentation and form of results can differ from one paper to another.

\subsection{Paper 1}
\subsection{Paper 2}
\subsection{Paper 3}
\subsection{Paper 4}
\subsection{Paper 5}
vergleich ergebnisse paper vs nachgebautem Code
vergleich sequential vs federated
kurz auf automatisches Federated eingehen und welche Ergebnisse damit erzielt werden können?

\section{Conclusion and Outlook}
This section will give a brief conclusion of results that we achieved and explore some ideas and concepts that can be used to further improve our automated federation test in the future.  

\subsection{Conclusion}
We were able to replicate results of all but one of the papers that we selected and successfully adopted their approaches to utilize the concept of federated learning. In almost all cases the federated version achieved similar results to the centralized one, indicating that federated learning can indeed be used in Covid research to preserve data privacy of sensitive medical data. The adaption was successful in CNN as well as GNN architectures, showing that not only one type of network can be used with federated learning. This could enable research projects with multiple entities, for example government health organizations, working together on one big machine learning model, which seemed to be impossible due to privacy concerns in the past.
We did not deploy our federation models on different hardware and thus completely ignored the communication usually necessary for aggregating local models. In addition, we used Federated Averaging, the most basic aggregation technique in federated learning, instead of a more sophisticated approach like weighted Federated Averaging or algorithms built on top of it like \enquote{FedProx}\cite{fed_prox} or \enquote{FedSplit}\cite{fed_split}. Nevertheless, our results still show that federated learning has a huge potential for Covid and other research and that further research with it should be done.

Further, we developed a python class which can be used to quickly and with relative ease test how federated learning performs with a given model and training methodology before having to commit to building a custom solution or integrate a complex framework into existing code for that purpose. This will hopefully make the barrier-of-entrance to more development with federated learning easier and result in more projects using it in the future.

\subsection{Outlook}
Even though our automated federation test works quite well on the models we tested it with we cannot guarantee that it works on any other. It should be tested with a lot more models and datasets to get a higher confidence in its ability to indeed work with most image based Pytorch models.  
Apart from more testing there is also a lot of room for potential improvement in other areas. First, the current data augmentations for increasing the data available on each client needs to be more configurable and make more augmentations available to fit more scenarios. Taking this a step further, we could also support non-image data and their transformations. For being able to test even more real-world scenarios, it would also be of value to be able to split and distribute the data in different, non-random ways (e.g. give some clients most of the data of one class and other clients most of the data of other classes).
By supporting Pytorch models we, at least in theory, also support models of frameworks that are based on Pytorch. While this is a big portion of all machine learning models, it is by no means all of them. Support for Tensorflow and its respective dataset format is almost a must if we want to make quick testing of federated learning available to a large portion of the machine learning community.


\cite{koethe}
\cite{covid_net}
\cite{dark_net}
\cite{dlh_net}
\cite{federated_machine_learning}
\cite{google_ai_federated_learning}
\cite{graph_covid_net}
\cite{tensorflow_federated}
\cite{pysyft}

\bibliographystyle{plain}
\bibliography{bibtex/references}

\end{document}
