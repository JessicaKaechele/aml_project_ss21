% This template was initially provided by Dulip Withanage.
% Modifications for the database systems research group
% were made by Conny Junghans,  Jannik Strötgen and Michael Gertz

\documentclass[
     12pt,         % font size
     a4paper,      % paper format
     BCOR10mm,     % binding correction
     DIV14,        % stripe size for margin calculation
     ]{article}

%%%%%%%%%%%%%%%%%%%%%%%%%%%%%%%%%%%%%%%%%%%%%%%%%%%%%%%%%%%%

% PACKAGES:

% Use German :
\usepackage[english]{babel}
% Input and font encoding
\usepackage[utf8]{inputenc}
\usepackage[T1]{fontenc}
\usepackage[title]{appendix}
% Index-generation
\usepackage{makeidx}
% Einbinden von URLs:
\usepackage{url}
% Special \LaTex symbols (e.g. \BibTeX):
%\usepackage{doc}
% Include Graphic-files:
\usepackage{graphicx}
% Include doc++ generated tex-files:
%\usepackage{docxx}

% Fuer anderthalbzeiligen Textsatz
\usepackage{setspace}

\usepackage{amsmath}
\usepackage{amssymb}
% F�r todos
\usepackage[obeyFinal]{easy-todo}

% hyperrefs in the documents
\PassOptionsToPackage{hyphens}{url}\usepackage[
  bookmarks=true,
  colorlinks,
  pdfpagelabels,
  pdfstartview = FitH,
  bookmarksopen = true,
  bookmarksnumbered = true,
  linkcolor = black,
  plainpages = false,
  hypertexnames = false,
  citecolor = black,
  urlcolor=black]{hyperref}
%\usepackage{hyperref}

\usepackage{csquotes}
\usepackage{tabularx}
\usepackage{subcaption}

% subfigures
\usepackage{subcaption}
\usepackage[font={small,it}]{caption}


%%%%%%%%%%%%%%%%%%%%%%%%%%%%%%%%%%%%%%%%%%%%%%%%%%%%%%%%%%%%

% OTHER SETTINGS:

% Choose language
\newcommand{\setlang}[1]{\selectlanguage{#1}\nonfrenchspacing}

% Written by comment
\newcommand{\comment}[1]{\vspace{-1em}\hspace{27pt}{\small\textit{#1}}\bigskip\par}
\newcommand{\subcomment}[1]{\vspace{-0.8em}\hspace{35pt}{\small\textit{#1}}\bigskip\par}
\newcommand{\subsubcomment}[1]{\vspace{-0.8em}\hspace{39pt}{\small\textit{#1}}\bigskip\par}
\newcommand{\paragraphcomment}[1]{\hspace{-8pt}{\small\textit{#1}}\hspace{8pt}}

\DeclareMathOperator*{\argmax}{arg\,max}
\DeclareMathOperator*{\argmin}{arg\,min}

\setcounter{tocdepth}{2} 

\begin{document}

% TITLE:
\pagenumbering{roman} 
\begin{titlepage}


\begin{center}
\textbf{ 
\Large Heidelberg University\\
\smallskip
}

\vspace{2cm}

\textbf{\large Project Report - Advanced Machine Learning}

\vspace{0.5\baselineskip}
{\huge
\textbf{Federated Machine Learning}
}
\end{center}

\vfill 

{\large
\begin{tabular}[l]{ll}
Team Member: & Jessica Kaechele, 3588787,\\ & MSc Applied Computer Science\\
  & Uo251@stud.uni-heidelberg.de\\
Team Member: & Jonas Reinwald, 3600238, \\ & MSc Applied Computer Science\\
  & am248@stud.uni-Heidelberg.de\\
\end{tabular}
}
\vspace*{2cm}

{
  \begin{center}
  \textbf{GitHub Repository:}
  \url{https://github.com/JessicaKaechele/aml_project_ss21}
  \end{center}
}

\vspace*{.5cm}

\end{titlepage} 

% \input{<file>}
\newpage
\tableofcontents

\newpage
\listoffigures
\listoftables

\newpage

\pagenumbering{arabic}

\begin{abstract}
    covid research and ML was asked for
    covid and privacy both highly aktuelle topics
    covid is bla
    privacy wegen bla
    what do they have in common / how can they be combined

    We evaluate die umsetzbarkeit und performance von federated learning and present a way to test it with almost no changes to existing code required
    motivated by the practical benefits of federated learning in the field of covid research

    Research nowadays is at an all-time high, with research papers getting published every day.
    Over the last decade alone, the number of research papers on machine learning has more than tripled \cite{dimensions}.
    However, this comes with a downside, as it is getting harder and harder to keep track of trends and find specific information.
    To address this issue, we propose a clustering solution that exploits the standardized submission pipeline.
    Each submission consists of a title, an abstract, a text body, and a set of keywords that best describe the paper.
    We use the keywords from a set of over two thousand research papers on machine learning as supervision for clustering the abstracts in the hopes of finding and exploring subtopics.
\end{abstract}

\section{Introduction}
\comment{Written by Jonas Reinwald}

The COVID-19 pandemic has been and is still a highly relevant topic concerning all of humanity. A lot of progress towards preventing its spread has been made in the last two years. But, while a vaccine is readily available in some countries, others are struggling to get a sufficient supply. And even if the vaccine was available everywhere, not every person is medically able to receive it. It is therefore important to keep researching and developing alternative methods of detection and prevention to keep people safe. New and improved machine learning based techniques focusing on different aspects of the disease are released at a steady pace, with one such field of study being the early and easy detection of COVID-19 from CT- or X-ray-scans to aid medical diagnosis in distinguishing it from normal pneumonia.

Privacy, and data privacy in particular, is another topic that, while it has been important for many decades, gets more attention and is put more into focus of peoples minds in recent years. Through the means of the internet, society as a whole releases more and more, often personal, information into the public domain. Massive amounts of data is usually needed and can be used for good by machine learning researches. But this data can also be misused and hurt those it comes from when used with the wrong intentions. Leaking and selling of compromised private data becomes more and more common the more data can be attained by attacking a single data collecting target. Even lists of only the worst data breaches are counting more than 40 entries for 2021 alone.\cite{data_breaches}
Medical data especially is usually regarded as highly sensitive and its storage or transfer is handled with a more careful approach. It is a reasonable request that people don't want their medical records leaked to the public. However, fast and accurate research into COVID-19 with machine learning is difficult if not enough data is available. 

The concept of federated learning could potentially help to achieve both data privacy and accurate research. By distributing the training of machine learning models onto multiple clients, there is no need to share data between them and other entities. As it is a relatively new approach, not many COVID-19 research papers exist that utilize it, making it hard to evaluate whether or not it is applicable in this field. We want to take a selection of papers working with image data, replicate their methods and results and apply federated learning to the approach in order to analyse the positive or negative impact in performance. In addition, we want to develop a means to automate this kind of test as much as possible for future COVID-19 research models. 

In this report we will explain the theory of federated machine learning and give an overview of the papers we used to test our federated learning approach with. We will then describe both the training of original and the federated process for each of these papers and discuss all of our results in relation to the authors original results. In the end there will be a brief summary of what we achieved and an outlook looking into improvements of our approach and additional research that could be done in the future.

Federated learning in general can also be used in other areas of research where privacy and computational power are a matter of concern.

\section{Methods}

no actual split of clients (only virtual as different models in same machine)

\subsection{Federated Learning}


\subsection{Paper replication}

\subsection{Methods of papers}

methods of other papers (focus on pytorch image data)
federated Learning
(maybe only in outlook) tensorflow-federated erwähnen?


\section{Development \& Training}
describe training and datasets for all models

\subsection{Custom ResNet}
- for testing / as replacement for original paper
- PySyft

zuerst paper angeschaut, dabei dsa problem gehabt, dass netzwerk nicht ausreichend beschrieben wurde
-> selbst eins gebaut und uns an die angaben so nah wie möglich gehalten

duplication, shuffling, batching

\subsection{Adapted Papers}

\paragraph{COVIDNet}
\paragraph{DLH-Covid}
\paragraph{DarkCovidNet}
\paragraph{GraphCovidNet}

erst andere paper nachgebaut
deren training prozess beschreiben - vergleich ergebnisse mit paper (in anderem kapitel!)
federated process 
  -> original paper nachstellen
  -> erst mit duplizierten daten
  -> dann mit split / augmented
  -> andere paper
  -> automatisiert


\subsection{Other Papers}


\subsection{Automated Federated Learning}
automation
same gpu resources as single model (but potentially more ram)
difference in time to train not taken into consideration because the communication overhead will most likely be way more noticeable (could be studied in the future)

\section{Experimental Results}
This section will show and compare results obtained from original approaches, our implementation of them and their adoption to federated learning. As mentioned in section \ref{sec:dev_and_training} we did not want to change more than absolutely necessary from the original implementation, including the testing and evaluation methods, which is why the presentation and form of results can differ from one paper to another.


\subsection{Federation Paper}\label{subsec:results_federation_paper}
For the \textit{federation paper} we did the most tests as we used it continuously during the project phase to improve our understanding and our implementation of the automated federation test. Table \ref{tab:results_federation_paper} shows the accuracy values obtained from different runs.

\begin{table}[htbp]
    \small
    \centering
    \caption{Comparison between federation paper\cite{federated_machine_learning} and our ResNet implementation.}
    \begin{tabular}{c|c|c|c|c}
        Model & Optimizer & \shortstack[c]{Data\\distribution} & Avg. method & Accuracy\\
        \hline
        FedPaper (cent.) & SGD & - & - & 0.9734 \\
        FedPaper (cent.) & Adam & - & - & 0.9664 \\
        FedPaper (fed.) & SGD & Duplicate & FedSGD & 0.9872\\
        FedPaper (fed.) & Adam & Duplicate & FedSGD & 0.9828\\
        ResNet (cent.) & Adam & - & - & 0.9649 \\
        ResNet (fed.) & Adam & Duplicate & FedSGD & 0.9886 \\
        ResNet (fed.) & Adam & Split & FedSGD & 0.9764 \\
        ResNet (fed.) & Adam & Split \& Augment & FedSGD & 0.9230 \\
        ResNet (fed.) & Adam & Split & FedAvg & 0.9596 \\
    \end{tabular}
    \label{tab:results_federation_paper}
\end{table}

The first four entries are taken from the paper itself and represent the baseline performance we wanted to match. As can be seen, our centralized learning implementation (row 5) has an accuracy less than 1\% worse than the best centralized value from the paper. 
The next row, which shows our federated ResNet with duplicated data on each client confirms the authors claim that the federation algorithm performs even better than the centralized one.
In real world scenarios the data is never duplicated on each client device, as this would defeat the purpose of federated learning.
This is why we also tested our implementation with an evenly, randomly split dataset on each client (row 7), which does not perform as good as the previous test but still better than the centralized model.
Row 8 shows the same test but with data that was additionally augmented to make the images less similar with each other. Curiously, this results in the worst performance of all runs. One explanation for this is that the model simply needs more epochs of training to learn the added variation in the data. Another explanation is that, because testing of this model was still done with un-augmented data, the results of the other models show some form of over-fitting to the given images which is not apparent in the current model.
The last row repeats the experiment from row 7, but uses Federated Averaging with 5 local epochs on each client for each global epoch instead of Federated SGD to aggregate the client model parameters. In theory this should perform better than its Federated SGD counterpart.

\subsection{COVID-Net}
The COVID-Net implementation from the authors had its own result metrics evaluation using sensitivity, specificity, positive predictive value and negative predictive value, the corresponding values can be seen in table \ref{tab:results_covidnet}. 

\begin{table}[htbp]
    \captionsetup[table]{justification=centering}
    \small
    \centering
    \caption{Comparison between COVID-Net paper\cite{covid_net} and our training of it.}
    \begin{tabular}{c|c|c|c|c}
        Model & Sensitivity & Specificity & \shortstack[c]{Positive\\predictive value}  & \shortstack[c]{Negative\\predictive value}\\
        \hline
        Original & 0.955 & 0.970 & 0.970 & 0.956 \\
        Our training (cent.) & 0.995 & 0.430 & 0.636 & 0.989 \\
        Our training (fed.) & - & - & - & -\\
    \end{tabular}
    \label{tab:results_covidnet}
\end{table}

As described in section \ref{subsubsec:dev_covidnet}, re-training of this model was not successful. Both specificity and positive predictive value are a lot worse than with the available pre-trained model. We did not attempt to implement a federated learning version because the results would not have been representative anyway if our centralized results are not at least close to the ones given by the authors.

\subsection{DLH-COVID}
The results for the DLH-COVID model depicted in table \ref{tab:results_dlh_covid} show that, with our centralized training, we could not quite match the metric scores taken from the paper, but our federated learning approach still performed better than the centralized one in every regard and is not that far off from the original results.

\begin{table}[htbp]
    \small
    \centering
    \caption{Comparison between DLH-COVID paper\cite{dlh_net} and our training of it.}
    \begin{tabular}{c|c|c|c|c}
        Model & Accuracy & Precision & Recall & F1-Score \\
        \hline
        Original & 0.967 & 0.965 & 0.958 & 0.961 \\
        Our training (cent.) & 0.947 & 0.950 & 0.928 & 0.939\\
        Our training (fed.) & 0.950 & 0.952 & 0.934 & 0.942\\
    \end{tabular}
    \label{tab:results_dlh_covid}
\end{table}

Additionally to the accuracy, precision, recall and f1-score the authors of the DLH-COVID paper also included a visual result evaluation in form of a heatmap, which can be seen in figure \ref{fig:dlh_all_heatmaps}.

\begin{figure}[htbp]
    \captionsetup[subfigure]{justification=centering}
    \centering
    \begin{subfigure}{.35\textwidth}
        \centering
        \includegraphics[width=\linewidth]{imgs/dlh_covid_paper_heatmap.png}
        \caption{DLH-COVID Paper}
        \label{fig:dlh_original_heatmap}
    \end{subfigure}

    \vspace{.3cm}
    \begin{subfigure}{.35\textwidth}
      \centering
      \includegraphics[width=\linewidth]{imgs/dlh_covid_own_seq_heatmap.png}
      \caption{Our DLH-COVID training}
      \label{fig:dlh_seq_heatmap}
    \end{subfigure}
    \begin{subfigure}{.35\textwidth}
      \centering
      \includegraphics[width=\linewidth]{imgs/dlh_covid_own_fed_heatmap.png}
      \caption{Our federated DLH-COVID training}
      \label{fig:dlh_fed_heatmap}
    \end{subfigure}
    \caption{Heatmap comparison between DLH-COVID paper\cite{dlh_net} and our training of it.}
    \label{fig:dlh_all_heatmaps}
\end{figure}

It shows the absolute values of actual COVID-19, Pneumonia and normal images on the left side and how many of those were predicted in which category on the bottom. Our federated training predicted slightly less disease images correct, but it was more often correct on normales images than the sequentially trained model.

\subsection{DarkCovidNet}
Table \ref{tab:results_darkcovidnet} contains the results for DarkCovidNet. Our training, both centralized and federated, was again not as good as the score reported in the paper. With a 5\% difference between the centralized models, it might be argued that the federated learning score is not really an indication of what a federated version of the authors training process could achieve, but because of the improvements we see in our training we think it could further improve the authors scores. This model is also the first and only one in our research where a federated training process performs worse than the model trained centrally.

\begin{table}[htbp]
    \small
    \centering
    \caption{Comparison between DarkCovidNet paper\cite{dark_net} and our training of it.}
    \begin{tabular}{c|c|c}
        Model & Avg. method & Accuracy (\%)\\
        \hline
        Original & - & 0.870 \\
        Our training (cent.) & - & 0.828\\
        Our training (fed.) & FedSGD & 0.764 \\
        Our training (fed.) & FedAvg & 0.838 \\
    \end{tabular}
    \label{tab:results_darkcovidnet}
\end{table}

In contrast to the results from section \ref{subsec:results_federation_paper} the Federated Averaging really did perform better than the SGD variant, and also better than our centralized model.

\subsection{GraphCovidNet}
The results from GraphCovidNet, which again include precision, recall, f1-score and accuracy and can be seen in table \ref{tab:results_graphcovidnet}, are the least and most interesting results at the same time. They are not really interesting because the authors achieved to build a model architecture which is able to perfectly predict every single test image and our own training matches these values as well. On the other hand, this model is the only one using an GNN architecture, which shows that federated learning in COVID research is not only useful for CNNs but other architecture types as well.

\begin{table}[htbp]
    \small
    \centering
    \caption{Comparison between GraphCovidNet paper\cite{graph_covid_net} and our training of it.}
    \begin{tabular}{c|c|c|c|c}
        Model & Precision & Recall & F1-Score & Accuracy \\
        \hline
        Original & 1.0 & 1.0 & 1.0 & 1.0\\
        Our training (cent.) & 1.0 & 1.0 & 1.0 & 1.0\\
        Our training (fed.) & 1.0 & 1.0 & 1.0 & 1.0\\
    \end{tabular}
    \label{tab:results_graphcovidnet}
\end{table}


\section{Conclusion and Outlook}
The last section will give a brief conclusion of results that we achieved and explore some ideas and concepts that can be used to further improve our automated federation test in the future.  

\subsection{Conclusion}
We were able to replicate results of all but one of the papers that we selected and successfully adopted their approaches to utilize the concept of federated learning. In almost all cases the federated version achieved similar results to the centralized one, indicating that federated learning can indeed be used in COVID-19 research to preserve data privacy of sensitive medical data. The adaption was successful in CNN as well as GNN architectures, showing that not only one type of network can be used with federated learning. This could enable research projects with multiple entities, for example government health organizations, working together on one big machine learning model, which seemed to be impossible due to privacy concerns in the past.
We did not deploy our federation models on different hardware and thus completely ignored the communication usually necessary for aggregating local models. In addition, we used Federated Averaging, the most basic aggregation technique in federated learning, instead of a more sophisticated approach like weighted Federated Averaging or algorithms built on top of it like \enquote{FedProx}\cite{fed_prox} or \enquote{FedSplit}\cite{fed_split}. Nevertheless, our results still show that federated learning has a huge potential for COVID-19 and other research and that further research with it should be done.

Further, we developed a python class which can be used to quickly and with relative ease test how federated learning performs with a given model and training methodology before having to commit to building a custom solution or integrate a complex framework into existing code for that purpose. This will hopefully make the barrier-of-entrance to more development with federated learning easier and result in more projects using it in the future.

\subsection{Outlook}
Even though our automated federation test works quite well on the models we tested it with we cannot guarantee that it works on any other. It should be tested with a lot more models and datasets to get a higher confidence in its ability to indeed work with most image based PyTorch models.  
Apart from more testing there is also a lot of room for potential improvement in other areas. First, the current data augmentations for increasing the data available on each client needs to be more configurable and make more augmentations available to fit more scenarios. Taking this a step further, we could also support non-image data and their transformations. For being able to test even more real-world scenarios, it would also be of value to be able to split and distribute the data in different, non-random ways (e.g. give some clients most of the data of one class and other clients most of the data of other classes).
By supporting PyTorch models we, at least in theory, also support models of frameworks that are based on PyTorch. While this is a big portion of all machine learning models, it is by no means all of them. Support for Tensorflow and its respective dataset format is almost a must if we want to make quick testing of federated learning available to a large portion of the machine learning community.


\bibliographystyle{plain}
\bibliography{bibtex/references}

\end{document}
